% !TeX spellcheck = en_US
% !TeX encoding = utf8
% !TeX program = xelatex
% !BIB program = bibtex
% \documentclass[mathserif,compress,12pt]{ctexbeamer}
\documentclass[12pt,notes,mathserif]{beamer}
% \documentclass[draft]{beamer}	
\usetheme{Singapore}
% \usetheme{Hannover}
%\usepackage{pgfpages}
%\setbeameroption{show notes on second screen}

\usepackage[british]{babel}
\usepackage{graphicx,hyperref,url}
% \usepackage{ru}
\usepackage{mmstyles,bm,ulem}

\usepackage{listings}
\usefonttheme[onlymath]{serif}
\usepackage{fontspec}
\usepackage{xeCJK}
% \pgfdeclareimage[width=\paperwidth,height=\paperheight]{bg}{background}
% \setbeamertemplate{background}{\pgfuseimage{bg}}
%% columns
\newcommand{\begincols}[1]{\begin{columns}{#1}}
\newcommand{\stopcols}{\end{columns}}
% \usepackage[backend=biber]{biblatex}
% \bibliography{./ref.bib}
%\addbibresource{ref.bib}
\usepackage{indentfirst}
\usepackage{longtable}
\usepackage{float}
%\usepackage{picins}
\usepackage{rotating}
\usepackage{subfigure}
\usepackage{tabu}
\usepackage{amsmath}
\usepackage{amssymb}
\usepackage{setspace}
\usepackage{amsfonts}
\usepackage{appendix}
\usepackage{listings}
\usepackage{xcolor}
\usepackage{colortbl}
\usepackage{geometry}
% \setCJKfamilyfont{cjkhwxk}{SimSun}
% \newcommand*{\cjkhwxk}{\CJKfamily{cjkhwxk}}
%\newfontfamily{\consolas}{Consolas}
%\newfontfamily{\monaco}{Monaco}
%\setmonofont[Mapping={}]{Consolas}	%英文引号之类的正常显示,相当于设置英文字体
%\setsansfont{Consolas} %设置英文字体 Monaco, Consolas,  Fantasque Sans Mono
% \setmainfont{Times New Roman}
% \newfontfamily{\consolas}{Times New Roman}
% \newfontfamily{\monaco}{Arial}
% \setCJKmainfont{Times New Roman}
%\setmainfont{MONACO.TTF}
%\setsansfont{MONACO.TTF}
\newcommand{\verylarge}{\fontsize{60pt}{\baselineskip}\selectfont}  
\newcommand{\chuhao}{\fontsize{44.9pt}{\baselineskip}\selectfont}  
\newcommand{\xiaochu}{\fontsize{38.5pt}{\baselineskip}\selectfont}  
\newcommand{\yihao}{\fontsize{27.8pt}{\baselineskip}\selectfont}  
\newcommand{\xiaoyi}{\fontsize{25.7pt}{\baselineskip}\selectfont}  
\newcommand{\erhao}{\fontsize{23.5pt}{\baselineskip}\selectfont}  
\newcommand{\xiaoerhao}{\fontsize{19.3pt}{\baselineskip}\selectfont} 
\newcommand{\sihao}{\fontsize{14pt}{\baselineskip}\selectfont}      % 字号设置  
\newcommand{\xiaosihao}{\fontsize{12pt}{\baselineskip}\selectfont}  % 字号设置  
\newcommand{\wuhao}{\fontsize{10.5pt}{\baselineskip}\selectfont}    % 字号设置  
\newcommand{\xiaowuhao}{\fontsize{9pt}{\baselineskip}\selectfont}   % 字号设置  
\newcommand{\liuhao}{\fontsize{7.875pt}{\baselineskip}\selectfont}  % 字号设置  
\newcommand{\qihao}{\fontsize{5.25pt}{\baselineskip}\selectfont}    % 字号设置 

\graphicspath{{./fig/}}

% \setbeamertemplate{footnote}{%
%   \hangpara{2em}{1}%
%   \makebox[2em][l]{\insertfootnotemark}\footnotesize\insertfootnotetext\par%
% }

\definecolor{cred}{rgb}{0.6,0,0}
\definecolor{cgreen}{rgb}{0.25,0.5,0.35}
\definecolor{cpurple}{rgb}{0.5,0,0.35}
\definecolor{cdocblue}{rgb}{0.25,0.35,0.75}
\definecolor{cdark}{rgb}{0.95,1.0,1.0}
\lstset{
	language=R,
	numbers=left,
	numberstyle=\tiny\color{black},
	keywordstyle=\color{cpurple}\consolas,
	commentstyle=\color{cgreen}\consolas,
	stringstyle=\color{cred}\consolas,
	frame=single,
	escapeinside=``,
	xleftmargin=1em,
	xrightmargin=1em, 
	backgroundcolor=\color{cdark},
	aboveskip=1em,
	breaklines=true,
	tabsize=3
} 

\providecommand{\tightlist}{%
  \setlength{\itemsep}{0pt}\setlength{\parskip}{0pt}}

  
% The title of the presentation:
%  - first a short version which is visible at the bottom of each slide;
%  - second the full title shown on the title slide;
% \title[]{\LARGE CSE 5526: Introduction to Neural Networks}

% Optional: a subtitle to be dispalyed on the title slide
\title{Support Vector Machines (SVMs), Part 2}

% The author(s) of the presentation:
%  - again first a short version to be displayed at the bottom;
%  - next the full list of authors, which may include contact information;
\author[YingmingLi]{Yingming Li \\ yingming@zju.edu.cn}
% The institute:
%  - to start the name of the university as displayed on the top of each slide
%    this can be adjusted such that you can also create a Dutch version
%  - next the institute information as displayed on the title slide

\institute[DSERC, ZJU]{Data Science \& Engineering Research Center, ZJU}
% Add a date and possibly the name of the event to the slides
%  - again first a short version to be shown at the bottom of each slide
%  - second the full date and event name for the title slide
\date[\today]{\today}

\begin{document}

\AtBeginSection[]
{
	\begin{frame}
		\frametitle{Outline}
		\tableofcontents[currentsection]
	\end{frame}
}

% \AtBeginSubsection[2-]
% {
%    \begin{frame}
%        \frametitle{Outline}
%        \tableofcontents[currentsection]
%    \end{frame}
% }
\begin{frame}[c]
	\titlepage
	\begin{center}
		Adapted from slides provided by Prof. Michael Mandel.
	\end{center}
\end{frame}

% 2
\begin{frame}[c]
\frametitle{Back to SVMs: Maximum margin solution is a fixed point of the Lagrangian function}
\begin{itemize}
\item Recall, the maximum margin hyperplane is argmin$_{\mathbf{w},b}||\mathbf{w}||^2$ subject to $d_p(\mathbf{w}^T\mathbf{x}_p+b)\geqslant{}1$
\begin{itemize}
\item Minimization of a quadratic function subject to multiple linear inequality constraints
\end{itemize}
\item Will use Lagrange multipliers, $a_p$, to write Lagrangian function

$L(\mathbf{w},b,\mathbf{a})=\dfrac{1}{2}||\mathbf{w}||^2-\sum\limits_pa_p(d_p(\mathbf{w}^T\mathbf{x}_p+b)-1)$
\item Note that $\mathbf{x}_p$ and $d_p$ are fixed for the optimization
\end{itemize}

\end{frame}


% 3
\begin{frame}[c]
\frametitle{Dual form of Lagrangian eliminates $\mathbf{w}$ and $b$}
\begin{itemize}
\item Set derivatives of $L(\mathbf{w},b,\mathbf{a})$ \wrt $\mathbf{w}$ and $b$ to 0

\begin{align*}
\dfrac{\partial}{\partial \mathbf{w}}L=0=\mathbf{w}-\sum_pa_pd_p\mathbf{x}_p\\
\Rightarrow \mathbf{w}=\sum_pa_pd_p\mathbf{x}_p\\
\dfrac{\partial}{\partial b}L=0=\sum_pa_pd_p
\end{align*}

\end{itemize}
\end{frame}

% 4
\begin{frame}[c]
\frametitle{Dual form of Lagrangian eliminates $\mathbf{w}$ and $b$}
\begin{itemize}
\item Note that:

$\mathbf{w}^T\mathbf{w}=\sum\limits_pa_pd_p\mathbf{w}^T\mathbf{x}_p=\sum\limits_p\sum\limits_qa_pa_qd_pd_q\mathbf{x}_p^T\mathbf{x}_q$
\item ``Primal'' form of Lagragian

$L(\mathbf{w},b,\mathbf{a})=\dfrac{1}{2}\mathbf{w}^T\mathbf{w}-\sum\limits_pa_p(d_p(\mathbf{w}^T\mathbf{x}_p+b)-1)$

$=\dfrac{1}{2}\mathbf{w}^T\mathbf{w}-\sum\limits_pa_pd_p\mathbf{w}^Tx_p-b\sum\limits_pa_pd_p+\sum\limits_pa_p$
\end{itemize}

\end{frame}

% 5
\begin{frame}[c]
\frametitle{Dual form of Lagrangian eliminates $\mathbf{w}$ and $b$}
$L(\mathbf{w},b,\mathbf{a})$

$=\dfrac{1}{2}\mathbf{w}^T\mathbf{w}-\sum\limits_pa_pd_p\mathbf{w}^T\mathbf{x}_p-b\sum\limits_pa_pd_p+\sum\limits_pa_p$

$=\left(\dfrac{1}{2}-1\right)\sum\limits_p\sum\limits_aa_pa_qd_pd_q\mathbf{x}_p^T\mathbf{x}_q-b\cdot 0+\sum\limits_pa_p$
\begin{itemize}
\item So dual form of Lagrangian:

$\tilde{L}(\mathbf{a})=-\dfrac{1}{2}\sum\limits_p\sum\limits_qa_pa_qd_pd_q\mathbf{x}_p^T\mathbf{x}_q+\sum\limits_pa_p$
\end{itemize}

\end{frame}

% 6
\begin{frame}[c]
\frametitle{Dual form of Lagrangian eliminates $\mathbf{w}$ and $b$}
\begin{itemize}
\item Dual form of Lagrangian, maximize:
𝑝
$\tilde{L}(\mathbf{a})=-\dfrac{1}{2}\sum\limits_p\sum\limits_qa_pa_qd_pd_q\mathbf{x}_p^T\mathbf{x}_q+\sum\limits_pa_p$
\item Subject to the constraints
\[
a_p\geqslant{}0, \quad \forall p\qquad \sum_p a_pd_p=0
\]
\item Another quadratic programming problem subject to linear inequality and equality constraints
\end{itemize}
\end{frame}



% 7
\begin{frame}[c]
\frametitle{Dual form allows use of Kernel function}
\begin{itemize}
\item In dual form, $\mathbf{x}_p$, $\vx_q$ only interact as inner products:
𝑝
$\tilde{L}(\mathbf{a})=-\dfrac{1}{2}\sum\limits_p\sum\limits_qa_pa_qd_pd_q\mathbf{x}_p^T\mathbf{x}_q+\sum\limits_pa_p$
\item Can replace $\mathbf{x}_p^T\mathbf{x}_q$ with kernel function $k(\mathbf{x}_p,\mathbf{x}_q)$
\item Think of kernel function as inner product of feature vector of $\mathbf{x}_p$s in some high dimensional space
\[
k(\mathbf{x}_p,\mathbf{x}_q)=\phi^T(\mathbf{x}_p)\phi(\mathbf{x}_q)
\]
\item But don't actually have to instantiate $\phi(\mathbf{x}_p)$
\begin{itemize}
\item More about kernels shortly
\end{itemize}
\end{itemize}
\end{frame}



% 8
\begin{frame}[c]
\frametitle{Dual form is faster to solve when $D>N$}
\begin{itemize}
\item Solving a quadratic program in $M$ variables takes takes $O(M^3)$ time in general
\item Primal form involves $D$ variables $(\mathbf{w})$
\begin{itemize}
\item Dimensionality of the data $\mathbf{x}_p$,
\item Or dimensionality of features of the data $\phi(\mathbf{x}_p)$
\end{itemize}
\item Dual form involves $N$ variables $(\mathbf{a})$
\begin{itemize}
\item Number of training points
\end{itemize}
\item SVMs are generally most useful with kernels
\begin{itemize}
\item So $D>N$ and the dual is faster to solve
\end{itemize}
\end{itemize}
\end{frame}


% 9
\begin{frame}[c]
\frametitle{Classify new points using $y(\mathbf{x})$}
\begin{itemize}
\item Actual prediction function is still
\[
y(\mathbf{x})=\mathbf{w}^T\mathbf{x}+b
\]
\item Get $\mathbf{w}$ from primal Lagrangian
\[
\mathbf{w}=\sum_pa_pd_p\mathbf{x}_p
\]
\item Will discuss $b$ shortly, so
\[
y(\mathbf{x})=\sum_pa_pd_p\mathbf{x}_p^T\mathbf{x}+b
\]
\end{itemize}
\end{frame}



% 10
\begin{frame}[c]
\frametitle{Classify new points using $y(\mathbf{x})$, with kernel}
\begin{itemize}
\item With a kernel, $\mathbf{w}^T=\sum_pa_pd_p\phi(\mathbf{x}_p)$
\item Actual prediction function is
\[
y(\mathbf{x})=\mathbf{w}^T\phi(\mathbf{x})+b
\]
\[
=\sum_pa_pd_p\phi^T(\mathbf{x}_p)\phi(\mathbf{x})+b
\]
\[
=\sum_pa_pd_pk(\mathbf{x}_p,\mathbf{x})+b
\]
\item In practice, save all $\mathbf{x}_p$ with $a_p> 0$
\begin{itemize}
\item And compute $k(\mathbf{x}_p,\mathbf{x})$ at test time
\end{itemize}
\end{itemize}
\end{frame}



% 11
\begin{frame}[c]
\frametitle{KKT Conditions}
\begin{itemize}
\item In the case of SVMs, the KKT conditions are
\[
a_p\geqslant{}0
\]
\[
d_py(\mathbf{x}_p)-1\geqslant{}0
\]
\[
a_p(d_py(\mathbf{x}_p)-1)=0
\]
\item So either $a_p=0$ or $d_py(\mathbf{x}_p)-1=0$
\begin{itemize}
\item Constraint from each point is either ignored or active
\end{itemize}
\item When $a_p=0$, $\mathbf{w}$ is independent of that point
\item When $d_py(\mathbf{x}_p)=1$, that point is on the margin
\begin{itemize}
\item It is a support vector
\item Thus only the support vectors contribute to $\mathbf{w}$
\end{itemize}
\end{itemize}
\end{frame}


% 12
\begin{frame}[c]
\frametitle{Compute $b$ from support vectors}
\begin{itemize}
\item Get $b$ from support vectors, which have margin 1
\item In the linear case, for a support vector $\mathbf{x}_q^s$
\[
y(\mathbf{x}_q^s)=d_p=\mathbf{w}^T\mathbf{x}_q^s+b
\]
\[
b=d_a^s -\sum_pa_pd_p\mathbf{x}_p^T\mathbf{x}_a^s
\]
\item When using a kernel
\[
b=d_a^s-\sum_pa_pd_pk(\mathbf{x}_p,\mathbf{x}_q^s)
\]
\item For numerical stability, average over all SVs
\end{itemize}
\end{frame}


% 13
\begin{frame}[c]
\frametitle{Summary so far}
\begin{itemize}
\item Finding the maximum margin hyperplane has been formulated as a constrained quadratic program
\begin{itemize}
\item Convex problem, well studied, easy conceptually to solve
\end{itemize}
\item Can be solved in the primal or dual formulation
\begin{itemize}
\item Dual formulation permits the use of kernel functions
\end{itemize}
\item Only some data points contribute to the solution
\begin{itemize}
\item The support vectors
\end{itemize}
\item So far, only applies to linearly separable data
\end{itemize}
\end{frame}



\begin{frame}
\begin{center}
\chuhao Thank you! %\fontspec{LHANDW.TTF}
\end{center}
\end{frame}
\end{document}
